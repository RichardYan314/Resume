%%%%%%%%%%%%%%%%%%%%%%%%%%%%%%%%%%%%%%%%%%%%%%%%%%%%%%%%%%%%%%%%%%%%%%%%%%%%%%%%
% Medium Length Graduate Curriculum Vitae
% LaTeX Template
% Version 1.2 (3/28/15)
%
% This template has been downloaded from:
% http://www.LaTeXTemplates.com
%
% Original author:
% Rensselaer Polytechnic Institute 
% (http://www.rpi.edu/dept/arc/training/latex/resumes/)
%
% Modified by:
% Daniel L Marks <xleafr@gmail.com> 3/28/2015
%
% Important note:
% This template requires the res.cls file to be in the same directory as the
% .tex file. The res.cls file provides the resume style used for structuring the
% document.
%
%%%%%%%%%%%%%%%%%%%%%%%%%%%%%%%%%%%%%%%%%%%%%%%%%%%%%%%%%%%%%%%%%%%%%%%%%%%%%%%%

%-------------------------------------------------------------------------------
%	PACKAGES AND OTHER DOCUMENT CONFIGURATIONS
%-------------------------------------------------------------------------------

%%%%%%%%%%%%%%%%%%%%%%%%%%%%%%%%%%%%%%%%%%%%%%%%%%%%%%%%%%%%%%%%%%%%%%%%%%%%%%%%
% You can have multiple style options the legal options ones are:
%
%   centered:	the name and address are centered at the top of the page 
%				(default)
%
%   line:		the name is the left with a horizontal line then the address to
%				the right
%
%   overlapped:	the section titles overlap the body text (default)
%
%   margin:		the section titles are to the left of the body text
%		
%   11pt:		use 11 point fonts instead of 10 point fonts
%
%   12pt:		use 12 point fonts instead of 10 point fonts
%
%%%%%%%%%%%%%%%%%%%%%%%%%%%%%%%%%%%%%%%%%%%%%%%%%%%%%%%%%%%%%%%%%%%%%%%%%%%%%%%%

\documentclass[margin]{res}  

% Default font is the helvetica postscript font
\usepackage{helvet}

\usepackage{fancyhdr}
\usepackage[us,12hr]{datetime} % `us' makes \today behave as usual in TeX/LaTeX
\fancypagestyle{plain}{
\fancyhf{}
\rfoot{Compiled on {\ddmmyyyydate\today} at \currenttime~EST}
\lfoot{Page \thepage}
\renewcommand{\headrulewidth}{0pt}}
\pagestyle{plain}


\usepackage{hyperref}
\hypersetup{
	breaklinks=true,
	colorlinks=true,
	urlcolor=blue
}

\expandafter\def\expandafter\UrlBreaks\expandafter{\UrlBreaks% save the current one
  \do\a\do\b\do\c\do\d\do\e\do\f\do\g\do\h\do\i\do\j%
  \do\k\do\l\do\m\do\n\do\o\do\p\do\q\do\r\do\s\do\t%
  \do\u\do\v\do\w\do\x\do\y\do\z\do\A\do\B\do\C\do\D%
  \do\E\do\F\do\G\do\H\do\I\do\J\do\K\do\L\do\M\do\N%
  \do\O\do\P\do\Q\do\R\do\S\do\T\do\U\do\V\do\W\do\X%
  \do\Y\do\Z\do\*\do\-\do\~\do\'\do\"\do\-}%

\newcommand{\fullhrulefill}{%
  \hspace*{-\sectionwidth}\hrulefill%
  }

% Increase text height
\textheight=700pt

\begin{document}

%-------------------------------------------------------------------------------
%	NAME AND ADDRESS SECTION
%-------------------------------------------------------------------------------
\name{Yunshan (Richard) Yan}

% Note that addresses can be used for other contact information:
% -phone numbers
% -email addresses
% -linked-in profile

\address{
Email : \href{mailto:richardyan314@foxmail.com}{richardyan314@foxmail.com}
\\LinkedIn : \url{https://www.linkedin.com/in/yunshan-richard-yan-58a2a617a}/
\\Github : \url{https://github.com/RichardYan314}
\\Blog : \url{https://richardyan314.github.io/}
\\
}

% Uncomment to add more addresses
%\address{Address 3 line 1\\Address 3 line 2\\Address 3 line 3}
%-------------------------------------------------------------------------------

\begin{resume}

%-------------------------------------------------------------------------------
%	EDUCATION SECTION
%-------------------------------------------------------------------------------

\fullhrulefill

\section{}
This file was compiled at 
{\shortdate\today} at \currenttime~EST.
Most up-to-date version can be accessed at: \\
\url{https://github.com/RichardYan314/Resume/blob/master/resume.pdf}

\section{}
A `.docx` formatted version generated from this file can be accessed at: \\
\url{https://github.com/RichardYan314/Resume/blob/master/resume.docx} \\
which may not be up to date.

\section{}
My unofficial transcript for both MASc and BASc degrees can be accessed at:\\
\url{https://github.com/RichardYan314/Resume/blob/master/transcript_unofficial.pdf}

% \section{OBJECTIVE}
% {\sl Pre-Final year B. Tech student at LNMIIT. Inquisitive, hard-working and consistent. Looking for internship opportunities at Microsoft where I can apply my skills and contribute to real-world projects }

\section{EDUCATION}
\textbf{Queen's University}, Canada\\
{\sl Master of Applied Science, Department of Electrical and Computer Engineering\\
Expected May 2019}

\textbf{Queen's University}, Canada\\
{\sl Bachelor of Applied Science, Department of Electrical and Computer Engineering\\
GPA: 
\hfill CGPA: 3.85/4.2}



%-------------------------------------------------------------------------------
%	COMPUTER SKILLS SECTION
%-------------------------------------------------------------------------------
\section{TECHNICAL\\SKILLS}

\textbf{Programming Languages : } Python, Java ,JavaScript, C,  C++, Haskell, Coq, Scheme, TXL
\\
\textbf{Database :} MySQL
\\
\textbf{General : } Embedded Systems, Data Mining, Algorithms.

%-------------------------------------------------------------------------------
% Modify the format of each position
\begin{format}
\title{l}\\
\dates{l}\location{r}\\
\body\\
\end{format}
%-------------------------------------------------------------------------------

\section{EXPERIENCE}

\textbf{Queen's University : Research Assistance \hfill{Sept. 18 - Dec. 18}\\}
\normalfont{Research Assistance in Software Reengineering Research Group.\\
url: http://post.queensu.ca/~zouy/ (The server does not have HTTPS capability)}

\textbf{Queen's University : Graduate Teaching Assistance \hfill{Sept. 18 - Dec. 18}\\}
\normalfont{Graduate TA for course ELEC 278: Data Structures. Duties involved supervising lab activities, proctoring quizzes, marking final exams, and answer student questions.}

\textbf{Queen's University : Undergraduate Teaching Assistance \hfill{Jan. 18 - Apr. 18}\\}
\normalfont{Undergraduate TA for course ELEC 274: Computer Architecture. Duties involved supervising lab activities and answer student questions.}

\textbf{Queen's University : Undergraduate Teaching Assistance \hfill{Sept. 17 - Dec. 17}\\}
\normalfont{Undergraduate TA for course ELEC 278: Data Structures. Duties involved supervising lab activities and answer student questions.}

\textbf{Queen's University : Undergraduate Teaching Assistance \hfill{Jan. 17 - Apr. 17}\\}
\normalfont{Undergraduate TA for course ELEC 274: Computer Architecture. Duties involved supervising lab activities and answer student questions.}

\textbf{Queen's University : Undergraduate Teaching Assistance \hfill{Sept. 16 - Dec. 16}\\}
\normalfont{Undergraduate TA for course ELEC 221: Electric Circuits. Duties involved supervising lab activities, preparing assignment solutions, and answer student questions.}

\section{PROJECTS}
\textbf{Networking Support for Terasic DE10 Board\hfill Sept. 2017 - Apr. 2018}

%\begin{position}
% \begin{itemize}
Networking capability was added to the Terasic DE10 board, a embedded system build around the Intel System-on-Chip(SoC) FPGA, by interfacing it to a WIZ812MJ, a commercially available network module, through both SPI and parallel bus communication interface.
This project is awarded as the Students' Choice: The Best Engineering Capstone Project.
% \end{itemize}
\begin{itemize}
\item \textbf{Technology/Tools:} FPGA, C, Nios II assembly, Altera Quartus II, TCP/IP stack.
\end{itemize}
%\end{position}

\textbf{FPGA Implementation of the MiniSRC processor\hfill Jan. - Apr. 2017}

%\begin{position}
% \begin{itemize}
The RISC style MiniSRC processor was implemented on Altera FPGA using VHDL language. In addition, an assembler was built to convert MiniSRC assembly language to Altera FPGA Memory Initialization File (.mif).
% \end{itemize}
\begin{itemize}
\item \textbf{Technology/Tools:} VHDL, FPGA, Altera Quartus II.
\end{itemize}
%\end{position}

\textbf{Compiler for Programming Language Garnet \hfill Sep - Dec 2016}

%\begin{position}
% \begin{itemize}
A compiler was developed for the programming language Garnet: a Ruby/Pascal like language. All four phases of the compiler (i.e., lexing, parsing, semantic analyzing, and code generating) are implemented in the S/SL language (Syntax/Semantic Language) (Holt, Richard C., James R. Cordy, and David B. Wortman. "An introduction to S/SL: Syntax/semantic language." ACM Transactions on Programming Languages and Systems (TOPLAS) 4.2 (1982): 149-178.)
% \end{itemize}
\begin{itemize}
\item \textbf{Technology/Tools:} S/SL, Linux Intel x86 assembly, Theories of Formal Languages and Automata.
\end{itemize}
%\end{position}


\section{CERTIFICATION}
\par
\normalfont{.}
\\
\normalfont{\textbullet{} \textbf{Machine Learning} by University of Stanford on \sl{Coursera}} \\
Verify : \url{https://www.coursera.org/account/accomplishments/certificate/QCXVU9KDVW59}
\\


\section{RELEVANT\\COURSES}
\par

\normalfont{\textbullet{} Digital Systems }
\normalfont{ \textbullet{} Data Structures}
\normalfont{ \textbullet{} Computer Architecture \\}
\normalfont{\textbullet{} Mechatronics Project   }
\normalfont{\textbullet{} Fundamentals of Software Development\\}
\normalfont{\textbullet{} Algorithms }
\normalfont{ \textbullet{} Probability \& Random Processes \\}
\normalfont{\textbullet{} Microprocessor Systems}
\normalfont{\textbullet{} Operating Systems}
\normalfont{\textbullet{} Software Specifications\\}
\normalfont{\textbullet{} Database Management Systems}
\normalfont{\textbullet{} Computer Graphics\\}
\normalfont{\textbullet{} Programming Language Processor}
\normalfont{\textbullet{} Digital Systems Engineering\\}
\normalfont{\textbullet{} Computer System Architecture}
\normalfont{\textbullet{} Computability \& Complexity\\}
\normalfont{\textbullet{} Formal Methods In Software Engineering}
\normalfont{\textbullet{} Image Processing \& Computer Vision\\}
\normalfont{\textbullet{} Number Theory \& Cryptography}
\normalfont{\textbullet{} Computer Networks\\}
\normalfont{\textbullet{} Control Of Discrete-Event Systems}
\normalfont{\textbullet{} Computational Complexity\\}
\normalfont{\textbullet{} Mining Software Engineering Data}
\normalfont{\textbullet{} Software Reengineering\\}
\normalfont{\textbullet{} Design recovery and Automated Evolution}
\normalfont{\textbullet{} Semantics of Programming Languages}

\section{Awards}

\normalfont{ \textbullet{} Queen's University Excellence Scholarship, 2014\\}
\normalfont{ \textbullet{} Dean's Scholar, 2015, 2016, 2017, 2018\\}
\normalfont{ \textbullet{} Ho Ming Tai Memorial Scholarship, 2015, 2016, 2017, 2018\\}
\normalfont{ \textbullet{} Teaching Assistant of the Year, 2017\\}
\normalfont{ \textbullet{} Students' Choice: The Best Engineering Capstone Project, 2018\\}

%-------------------------------------------------------------------------------


\end{resume}
\(\)\end{document}